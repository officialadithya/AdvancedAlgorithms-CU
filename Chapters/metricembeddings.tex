\section{Lecture 14: March 4, 2025}

    \subsection{Introduction to Metric Embeddings}

        Recall the following definitions of metrics and metric spaces; we defer unfamilar readers to \cite{rudin1976principles,lang1997analysis}.
        \begin{definition}{\Stop\,\,Metrics}{metrics}
            
            Let \(X\neq\emptyset\) be a set; a metric on \(X\) is a function \(d:X\times X\to\mathbb{R}\) such that for all \(x,y,z\in X\),
            \begin{enumerate}
                \item[(M1)] \(d(x,y)\geq0\),
                \item[(M2)] \(d(x,y)=0\iff x=y\),
                \item[(M3)] \(d(x,y)=d(y,x)\), and
                \item[(M4)] \(d(x,y)\leq d(x,z)+d(z,y)\).
            \end{enumerate}

        \end{definition}
        \begin{definition}{\Stop\,\,Metric Spaces}{metricspaces}
            Given a metric \(d\) on a set \(X\neq\emptyset\), we call \((X,d)\) a metric space.
        \end{definition}
        \begin{remark*}
            Given any norm \(||\cdot||:V\to\mathbb{F}\) on a normed space \(V\), the function \(d\) with \((x,y)\mapsto ||x-y||\) is a metric.
        \end{remark*}
        \pagebreak
        \vphantom
        \\
        \\
        We now introduce metric embeddings.
        \begin{definition}{\Stop\,\,Metric Embeddings and Distortion}{metricembeddings}
            Let \((X,d_X)\) and \((Y,d_Y)\) be metric spaces. An injective function \(f:X\to Y\) is called a metric embedding. The distortion of \(f\) is 
            \begin{equation*}
                \inf_{d\geq 1}\{\exists r>0:\forall x,y\in X, rd_X(x,y)\leq d_Y(f(x),f(y))\leq drd_X(x,y)\}.
            \end{equation*}
        \end{definition}